\documentclass[12pt,letterpaper]{article}
\usepackage[utf8]{inputenc}
\usepackage[spanish,USenglish]{babel}
\usepackage{amsmath,amsfonts,amssymb,amsthm}
\usepackage{graphicx}
\usepackage{epsfig}
\usepackage{setspace}
\usepackage{enumerate} 
%gráficos y figuras
\usepackage{pgf,tikz,pgfplots}
\usetikzlibrary{arrows}
\pgfplotsset{compat=1.15}
\usetikzlibrary{trees,arrows,positioning,calc}
\tikzstyle{redVertex}  =[draw,fill=red,circle,minimum size=30pt,inner sep=-0pt, text=white]
\tikzstyle{blackVertex}=[draw,fill=black,circle,minimum size=30pt,inner sep=0pt, text=white]
\tikzstyle{nil}=[draw,fill=black,rectangle,minimum size=30pt,inner sep=0pt, text=white]
%escribir programas
\usepackage{listings}
%encabezado
\usepackage{fancyhdr}
\pagestyle{fancy}
\fancyhf{}
% Números de página en las esquinas de los encabezados
\fancyhead[R]{\thepage} 
%Espacio para Titulo (revisar warnings)
\setlength{\headheight}{14.5pt}
% Formato para la sección: N.M. Nombre
\renewcommand{\sectionmark}[1]{\markright{\textbf{\thesection. #1}}{}} 
%título
\title{ \textbf{Tarea Dos} \\ Optimización I}
\author{Esteban Reyes Saldaña}
\date{\today}
%definiciones
\theoremstyle{definition}
\newtheorem{problm}{Problema}
\usepackage{colortbl}
\usepackage{tabularx}
\usepackage{dcolumn}
\usepackage{multirow}
\usetikzlibrary[patterns]

\begin{document}
	
\selectlanguage{spanish}
\maketitle 
\begin{problm}
	La derivada direccional $ \dfrac{\partial f}{\partial v} (x_0, y_0, z_0) $ de una función diferenciable $ f $ son $ \dfrac{3}{\sqrt{2}}, \dfrac{1}{\sqrt{2}} $ y $ -\dfrac{1}{\sqrt{2}} $ en las direcciones de los vectores $ \left[ 0, \dfrac{1}{\sqrt{2}}, \dfrac{1}{\sqrt{2}} \right]^T $, $ \left[ \dfrac{1}{\sqrt{2}}, 0, \dfrac{1}{\sqrt{2}} \right]^T $ y $ \left[ \dfrac{1}{\sqrt{2}}, \dfrac{1}{\sqrt{2}}, 0 \right]^T $. Calcule $ \nabla f(x_0, y_0, z_0) $,
	\\
	\textbf{Solución}. En clase se vio que dada  una función diferenciable $ f: \mathbb{R}^n \to \mathbb{R} $ entonces
	\[ D_a f(x, y, z) = \nabla f(x,y,z)^T a. \]
	Sean 
	\begin{eqnarray*}
		u & = & \left[ 0, \dfrac{1}{\sqrt{2}}, \dfrac{1}{\sqrt{2}} \right]^T  \\
		v & = & \left[ \dfrac{1}{\sqrt{2}}, 0, \dfrac{1}{\sqrt{2}} \right]^T \\
		w & = & \left[ \dfrac{1}{\sqrt{2}}, \dfrac{1}{\sqrt{2}}, 0 \right]^T.
	\end{eqnarray*}
	y 
	\[ \nabla f(x_0,y_0,z_0) = \left[ \dfrac{\partial f}{\partial x}(x_0,y_0, z_0), \dfrac{\partial f}{\partial y}(x_0,y_0, z_0), \dfrac{\partial f}{\partial z}(x_0,y_0, z_0) \right]^T \]
	Luego,
	\begin{eqnarray*}
		\dfrac{3}{\sqrt{2}} = \dfrac{\partial f}{\partial u} (x_0, y_0, z_0) & = & \nabla f(x_0, y_0, z_0)^T u \\
		\dfrac{1}{\sqrt{2}} = \dfrac{\partial f}{\partial v} (x_0, y_0, z_0) & = & \nabla f(x_0, y_0, z_0)^T v \\
		-\dfrac{1}{\sqrt{2}} = \dfrac{\partial f}{\partial w} (x_0, y_0, z_0) & = & \nabla f(x_0, y_0, z_0)^T w.
	\end{eqnarray*}
	Realizando el producto punto entre el gradiente y los vectores $ u,v,w $ se obtiene el sistema
	\begin{eqnarray*}
		\dfrac{1}{\sqrt{2}} \cdot \dfrac{\partial f}{\partial y}(x_0,y_0, z_0) + \dfrac{1}{\sqrt{2}} \cdot \dfrac{\partial f}{\partial z}(x_0,y_0, z_0) & = & \dfrac{3}{\sqrt{2}} \\
		\dfrac{1}{\sqrt{2}} \cdot \dfrac{\partial f}{\partial x}(x_0,y_0, z_0) + \dfrac{1}{\sqrt{2}} \cdot \dfrac{\partial f}{\partial z}(x_0,y_0, z_0) & = & \dfrac{1}{\sqrt{2}} \\
		\dfrac{1}{\sqrt{2}} \cdot \dfrac{\partial f}{\partial x}(x_0,y_0, z_0) + \dfrac{1}{\sqrt{2}} \cdot \dfrac{\partial f}{\partial y}(x_0,y_0, z_0) & = & -\dfrac{1}{\sqrt{2}}
	\end{eqnarray*}
	que es lineal respecto a las derivadas parciales en las direcciones canónicas. Así que se procede a resolver el sistema matricial asociado
	\begin{eqnarray*}
		\left( \begin{matrix}
			           0         & \dfrac{1}{\sqrt{2}} & \dfrac{1}{\sqrt{2}} & | & \dfrac{3}{\sqrt{2}} \\
			 \dfrac{1}{\sqrt{2}} &         0           & \dfrac{1}{\sqrt{2}} & | & \dfrac{1}{\sqrt{2}} \\
			 \dfrac{1}{\sqrt{2}} & \dfrac{1}{\sqrt{2}} &        0            & | & -\dfrac{1}{\sqrt{2}} \\
		\end{matrix} \right) & \sim &
		\left( \begin{matrix}
			0 & 1 & 1 & | & 3 \\
			1 & 0 & 1 & | & 1 \\
			1 & 1 &  0 &| & -1 
		\end{matrix} \right) \\
							& \sim &
							\left( \begin{matrix}
								0 & 1 &  1 & | & 3 \\
								1 & 0 &  1 & | & 1 \\
								1 & 0 & -1 & | & -4 
							\end{matrix} \right) \\
						    & \sim &
						    \left( \begin{matrix}
						    	0 & 1 & 1 & | & 3 \\
						    	1 & 0 & 1 & | & 1 \\
						    	2 & 0 & 0 & | & -3 
						    \end{matrix} \right) \\
					        & \sim &
					        \left( \begin{matrix}
					        	0 & 1 & 1 & | & 3 \\
					        	1 & 0 & 1 & | & 1 \\
					        	1 & 0 & 0 & | & -3/2 
					        \end{matrix} \right) \\
				            & \sim &
			                \left( \begin{matrix}
			                	0 & 1 & 0 & | & 1/2 \\
			                	0 & 0 & 1 & | & 5/2 \\
			                	1 & 0 & 0 & | & -3/2 
			                \end{matrix} \right).
	\end{eqnarray*}
	De lo anterior, se concluye que
	\[ \nabla f (x_0, y_0, z_0) = \left[ -\dfrac{3}{2}, \dfrac{1}{2}, \dfrac{5}{2} \right]^T. \]
\end{problm}

\begin{problm}
	Demuestre que las curvas de nivel de la función $ f(x,y) = x^2 + y^2 $ son ortogonales a las curvas de nivel de $ g(x,y) = \dfrac{y}{x} $ para todo $ (x,y) $.
	\begin{proof}
		Observemos que dada una función diferenciable  $ f: \mathbb{R}^n \to \mathbb{R} $, se tiene que $ \nabla f $ es siempre ortogonal a la gráfica de dicha función. Así, dada $ f(x,y) $ diferenciable, $ \nabla f(x,y) $ es ortogonal a la gráfica de la función, es decir, es ortogonal a toda curva de nivel. Luego, para probar la ortogonalidad entre curvas de nivel entre $ f(x,y) $ y $ g(x,y) $ se probará la ortogonalidad entre los gradientes.
		\begin{eqnarray*}
			\nabla f(x,y) & = & \left[ 2x, 2y \right]^T \\
			\nabla g(x,y) & = & \left[ -\dfrac{y}{x^2}, \dfrac{1}{x} \right]^T
		\end{eqnarray*}
		Entonces,
		\begin{eqnarray*}
			<\nabla f(x,y), \nabla g(x,y) > & = & -\dfrac{2xy}{x^2} + \dfrac{2y}{x} \\
											& = & -\dfrac{2y}{x} + \dfrac{2y}{x} \\
											& = & 0.
		\end{eqnarray*}
	\end{proof}
\end{problm}

\begin{problm}
	Calcule los puntos estacionarios de $ f(x,y) = \dfrac{3x^4 - 4x^3 - 12x^2 + 18}{12(1+4y^2)} $ y determine su tipo correspondiente (máximo, mínimo o punto silla).
	\\
	\textbf{Solución.} Buscamos los puntos tales que 
	\[ \nabla f(x,y) = (0,0). \]
	Tenemos que
	\begin{eqnarray*}
		\dfrac{\partial}{\partial x} f(x,y) & = & \dfrac{12(x^3 - x^2 -2x)}{12(1+4y^2)} \\
											& = & \dfrac{x(x^2 - x -2)}{1+4y^2}
	\end{eqnarray*}
	\begin{eqnarray*}
		\dfrac{\partial}{\partial y} f(x,y) & = & \dfrac{3x^4 - 4 x^3 - 12x^2 +18}{12} \cdot \dfrac{-8y}{(1+4y^2)^2}\\
		& = & - \dfrac{2(3x^4 - 4 x^3 - 12x^2 +18) y}{3(1+4y^2)^2}
	\end{eqnarray*}
	De donde $ \nabla f(x,y) = 0 $ sí y solo si
	\begin{eqnarray}
		x(x^2-x-2) = x(x-2)(x+1) & = & 0 \\
		(3x^4 - 4 x^3 - 12x^2 +18) y & = & 0.
	\end{eqnarray}
	De la ecuación (1) tenemos que
	\[ x = -1 \textup{ o } x = 0 \textup{ o } x = 2. \]
	Sustituyendo estos valores en la ecuación (2) tenemos que
	\\
	para $  x = -1 $,
	\begin{eqnarray*}
		(3(-1)^4 - 4 (-1)^3 - 12 (-1)^2 + 18) y & = & 0 \\
		(3+4-12+18) y & = & 0 \\
		13 y & = & 0 \\
		 y & = & 0.
	\end{eqnarray*}
	Para $ x = 0 $,
	\begin{eqnarray*}
		(3(0) - 4(0) - 12(0) + 18) y & = & 0 \\
		18 y & = & 0 \\
		y & = & 0.
	\end{eqnarray*}
	Para $ x = 2 $, 
	\begin{eqnarray*}
		(3(2)^4 - 4 (2)^3 - 12 (2)^2 + 18) y & = & 0 \\
		-14 y & = & 0 \\
		y & = & 0.
	\end{eqnarray*}
	Así que los \textbf{puntos estacionarios} son
	\[ (-1,0), (0, 0) \textup{ y } (2,0). \]
	Para identificar el tipo de punto estacionario se observarán los valores propios de la matriz Hessiana asociada a dichos puntos. Obteniendo el Hessiano,
	\begin{eqnarray*}
		\dfrac{\partial^2 f(x,y) }{\partial x^2} & = & 3\dfrac{3x^2 - 2x -2}{1+4y^2} \\
		\dfrac{\partial^2 f(x,y) }{\partial y \partial x} & = & \dfrac{x(x^2 -x-2)}{(1+4y^2)^2} (-8y) \\
												 & = & -\dfrac{8xy(x^2 -x-2)}{(1+4y^2)^2} \\
		\dfrac{\partial^2 f(x,y) }{\partial x \partial y} & = & -\dfrac{2}{3} \cdot \dfrac{(12x^3 - 12x^2 -24x)y}{(1+4y^2)^2} \\
		\dfrac{\partial^2 f(x,y) }{\partial y^2} & = & -\dfrac{2}{3}(3x^4 - 4x^3 - 12x^2 + 18) \left( - \dfrac{2}{(1+4y^2)^3} (8y) + \dfrac{1}{(1+4y^2)^2}  \right) \\
		& = & -\dfrac{2}{3}(3x^4 - 4x^3 - 12x^2 + 18) \left( \dfrac{-16y +1 +4y^2}{(1+4y^2)^3} \right). \\									 
	\end{eqnarray*}
	Entonces
	\begin{eqnarray*}
		H_f(x,y) & = & \left(\begin{matrix}
							\dfrac{3x^2 - 2x -2}{1+4y^2} & -\dfrac{8xy(x^2 -x-2)}{(1+4y^2)^2} \\
							-\dfrac{2y(12x^3 - 12x^2 -24x)}{3(1+4y^2)^2} & 
							-\dfrac{2 (3x^4 - 4x^3 - 12x^2 + 18)(4y^2 - 16y + 1) }{3(1+4y^2)^3}.
		\end{matrix}\right)
	\end{eqnarray*}
	Luego,
	\begin{eqnarray*}
		H_f(-1,0) & = & \left(\begin{matrix}
			\dfrac{3+2-2}{1} & 0 \\
			0 & 
			-\dfrac{2 (3 + 4 - 12 + 18)(1) }{1}.
		\end{matrix}\right) \\
				  & = & \left(\begin{matrix}
				  	3 & 0 \\
				  	0 & 
				  	-\dfrac{26}{3}.
				  \end{matrix}\right)
	\end{eqnarray*}

	\begin{eqnarray*}
		H_f(0,0) & = & \left(\begin{matrix}
			-2 & 0 \\
			0 & 
			-\dfrac{2 (18)}{3}.
		\end{matrix}\right) \\
		& = & \left(\begin{matrix}
			-2 & 0 \\
			0 & 
			-12.
		\end{matrix}\right)
	\end{eqnarray*}

	\begin{eqnarray*}
		H_f(2,0) & = & \left(\begin{matrix}
			3(4) - 2(2) -2 & 0  \\
			0 & 
			-\dfrac{2 (3(16) - 4(8) - 12(4) +18)}{3}.
		\end{matrix}\right) \\
		& = & \left(\begin{matrix}
			 6 & 0 \\
			 0 & -\dfrac{56}{3}.
		\end{matrix}\right)
	\end{eqnarray*}
	Dado que las matrices Hessianas asociadas son diagonales, entonces sus eigenvalores son los valores en la diagonal. Dichos valores son reales, entonces, por un teorema visto en clase, concluímos que
	\\
	$ (-1,0) $ es un \textbf{punto silla} (dado que tiene eigenvalores tanto positivos como negativos).
	\\
	$ (0,0) $ es un \textbf{mínimo} (dado que todos sus eigenvalores son negativos).
	\\
	$ (2,0) $ es un \textbf{punto silla} (dado que tiene eigenvalores tanto positivos como negativos).
\end{problm}

\begin{problm}
	Calcule el gradiente $ \nabla f(x) $ y el Hessiano $ \nabla^2 f(x) $ de la función de Rosenbrock
	\[ f(x) = \sum_{i = 1}^{N-1} [100(x_{i+1} - x_i^2)^2 + (1 - x_i)^2] \]
	donde $ x = [ x_1, \dots, x_N ]^T \in\mathbb{R}^N $.
	\\
	\textbf{Solución}.  Primero notemos que  para $ 2 \leq j \leq N-1 $ se puede reescribir $ f(x) $ como
	\begin{eqnarray*}
		f(x) & = & \sum_{i = 1}^{j-2}  [100(x_{i+1}- x_i^2)^2 + (1-x_i)^2] + [100(x_{j} - x_{j-1}^2)^2 + (1-x_{j-1})^2] \\
			 &   & [100(x_{j+1} - x_{j}^2)^2 + (1-x_{j})^2] +  \sum_{i = j+1}^{N-1} [100(x_{i-1} - x_i^2)^2 - (1-x_i)^2 ]. \\
	\end{eqnarray*}
	Notemos que el primer y útlimo término son cosntantes con respecto a $ x_j $, entonces se anularán al calcular la correspondiente derivada parcial.
	\\
	Para $ D_1 $ se tiene que
	\begin{eqnarray*}
		\dfrac{\partial }{\partial x_1} f(x) & = & 100(2)(x_2 - x_1^2) (-2x_1) - 2(1-x_1)  \\
										     & = &  -400(x_2 - x_1^2)x_1 - 2(1-x_1).
	\end{eqnarray*}
	Para $ D_j \in\{ 2, \dots, N-1 \} $,
	\begin{eqnarray*}
		\dfrac{\partial }{\partial x_j} f(x) & = & [200(x_j - x_{j-1}^2)] + [ 200(x_{j+1} - x_j^2)(-2x_j)] - [2(1-x_j) ] \\
		   									 & = & 200(x_j - x_{j-1}^2) - 400 (x_{j+1} - x_j^2) x_j - 2(1-x_j)
	\end{eqnarray*}
	Para $ D_N $, 
	\begin{eqnarray*}
		\dfrac{\partial }{\partial x_N} f(x) & = & 200(x_N - x_{N-1}^2).
	\end{eqnarray*}
	De lo anterior tenemos que
	\begin{equation}
		\nabla f(x)  =  \left[\begin{matrix}
								-400(x_2 - x_1^2) x_1 - 2(1-x_1) \\
								\vdots							 \\
								200(x_j - x_{j-1}^2) - 400(x_{j+1} - x_j^2)x_j - 2(1-x_j)					 \\
								\vdots  						 \\
								200(x_N - x_{N-1}^2).
		\end{matrix}\right]
	\end{equation}
	\textbf{Nota}. Para calcular el Hessiano,se debe considerar como casos particulares, aquellos que se apliquen a $ \dfrac{\partial f(x) (x)}{\partial x_1} $ y $ \dfrac{\partial f(x) (x)}{\partial x_N} $ (como se observó en el gradiente). 
	\\
	Para $ D_{x_k x_1 } $,
	\begin{eqnarray*}
		\dfrac{\partial^2 f(x)}{\partial x_k \partial x_1} = \left\{\begin{matrix}
																	1200x_1^2 -400 x_2 +2 & si & k = 1 \\
																	-400x_1				  & si & k = 2 \\
																	0					  & e.o.c &
															 \end{matrix}\right.
	\end{eqnarray*}
	Para $ D_{x_k x_j} $ con $ j \in\{ 2,\dots, N-1 \} $,
	\begin{eqnarray*}
		\dfrac{\partial^2 f(x)}{\partial x_k \partial x_j} = \left\{\begin{matrix}
			1200x_j^2 -400 x_2 +2 & si & k = j   \\
			-400x_j				  & si & k = j+1 \\
			-400 x_{j-1}          & si & k = j-1 \\
			0					  & e.o.c &
		\end{matrix}\right.
	\end{eqnarray*}
	Para $ D_{x_k x_N} $,
	\begin{eqnarray*}
		\dfrac{\partial^2 f(x)}{\partial x_k \partial x_1} = \left\{\begin{matrix}
			-400 x_{N-1} & si & k = N - 1 \\
			200		     & si & k = N \\
			0					  & e.o.c &
		\end{matrix}\right.
	\end{eqnarray*}
	Así que el Hessiano tiene la forma
	\begin{eqnarray}
		H(x) = \left( \begin{matrix}
							1200x_1^2 -400 x_2 +2  & -400 x_1 & 0 & 0 & \dots & 0 & 0 \\
							-400 x_1 & 1200x_2^2 - 400 x_3 + 202 & -400 x_2 & 0 \dots & 0 & 0 \\
							\vdots & \vdots & \vdots & \vdots & \ddots & \vdots & \vdots \\
							0 & 0 & 0 & 0 & \dots & -400x_{N-1} & 200
		\end{matrix} \right).
	\end{eqnarray}
\end{problm}

\begin{problm}
	Demuestre, sin usar condiciones de optimalidad, que $ f(x) > f(x^*) $ para todo $ x \neq x^* $ si
	\[ f(x) = \dfrac{1}{2} x^T Q x - b^T x \]
	$ Q = Q^T > 0 $ y $ Q x^* = b $.
	\begin{proof}
		Primero observemos que dados $ x,y\in\mathbb{R}^n $ tenemos que, como $ Q = Q^T $,
		\[ x^T Q y = (Qy)^T x = y^T Q^T x = y^T Q x. \]
		Por lo que
		\begin{equation}\label{propiedad}
			x^T Q y = y^T Q x.
		\end{equation}
		
		Ahora,
		\begin{eqnarray*}
			f(x) & = & f(x^* + (x - x^*)) \\
				& = & \dfrac{1}{2} (x^* + (x - x^*))^T Q (x^* + (x - x^*)) - b^T (x^* + (x - x^*)) \\
				& = & \dfrac{1}{2} \left(  (x^*)^T Q + (x - x^*)^T Q \right) (x^* + (x - x^*)) - b^T (x^* + (x - x^*)) \\
				& = & \dfrac{1}{2} (x^*)^T Q x^* + \dfrac{1}{2} (x^*)^T Q (x - x^*) + \dfrac{1}{2} (x - x^*)^T Q x^* \\
				&   & + \dfrac{1}{2} (x - x^*)^T Q (x - x^*) -b^T(x^* + (x - x^*)) \\
				& = & \dfrac{1}{2} (x^*)^T Q x^* +\underbrace{ \dfrac{1}{2} (x - x^*)^T Q x^*}_{(\ref{propiedad})} + \dfrac{1}{2} (x - x^*)^T Q x^* \\
				&   & + \dfrac{1}{2} (x - x^*)^T Q (x - x^*) -b^T(x^* + (x - x^*))
			\end{eqnarray*}
	
	
	
	
	
	
		Ahora,
		\begin{eqnarray*}
			     & = & \dfrac{1}{2} (x^*)^T Q x^* + (x - x^*)^T Q x^* + \underbrace{\dfrac{1}{2} (x - x^*)^T Q (x - x^*)}_{\textbf{Definida Positiva}} \\
			     &   & - b^T(x^* + (x - x^*)) \\
			     & > & \dfrac{1}{2} (x^*)^T Q x^* + (x - x^*)^T Q x^* - b^T(x^* + (x - x^*))  \\
			     & = & \dfrac{1}{2} (x^*)^T Q x^* - b^T x^* + (x - x^*)^T Q x^* - b^T(x - x^*)  \\
			     & = & f(x^*) + (x - x^*)^T Q x^* - (x - x^*)^T b  \\
			     & = & f(x^*) + (x - x^*)^T (Q x^* - b)  \\
			     & = & f(x^*) + (x - x^*)^T (\underbrace{b}_{\textup{hipótesis}} - b)  \\
			     & = & f(x^*)
		\end{eqnarray*}
		Así que $ f(x) > f(x^*) $.
	\end{proof}
\end{problm}

\end{document}
